\input{snippet/beamer_head}
\begin{document}

%\includeonlyframes{current}

\logo{\includegraphics[height=0.08\textwidth]{qr.png}}

% 在每个Section前都会加入的Frame
\AtBeginSection[]
{
  \begin{frame}<beamer>
    %\frametitle{Outline}
    \frametitle{教学提纲}
    \setcounter{tocdepth}{3}
    \begin{multicols}{2}
      \tableofcontents[currentsection,currentsubsection]
      %\tableofcontents[currentsection]
    \end{multicols}
  \end{frame}
}
% 在每个Subsection前都会加入的Frame
\AtBeginSubsection[]
{
  \begin{frame}<beamer>
%%\begin{frame}<handout:0>
%% handout:0 表示只在手稿中出现
    \frametitle{教学提纲}
    \setcounter{tocdepth}{3}
    \begin{multicols}{2}
    \tableofcontents[currentsection,currentsubsection]
    \end{multicols}
%% 显示在目录中加亮的当前章节
  \end{frame}
}

% 为当前幻灯片设置背景
%{
%\usebackgroundtemplate{
%\vbox to \paperheight{\vfil\hbox to
%\paperwidth{\hfil\includegraphics[width=2in]{tijmu_charcoal.png}\hfil}\vfil}
%}
\begin{frame}[plain]
  \begin{center}
    {\Huge 故事中的统计学\\}
    \vspace{1cm}
    {\LARGE 天津医科大学\\}
    %\vspace{0.2cm}
    {\LARGE 生物医学工程与技术学院\\}
    \vspace{1cm}
    {\large 2016-2017学年下学期(春)\\ 公共选修课}
  \end{center}
\end{frame}
%}



%\includeonlyframes{current}

\title[案例集锦与数据反驳]{第六章\quad 案例集锦与数据反驳}
\author[Yixf]{伊现富(Yi Xianfu)}
\institute[TIJMU]{天津医科大学(TIJMU)\\ 生物医学工程与技术学院}
\date{2018年4月}

\begin{frame}
  \titlepage
\end{frame}

\begin{frame}[plain,label=current]
  \frametitle{教学提纲}
  \setcounter{tocdepth}{3}
  \begin{multicols}{2}
    \tableofcontents
  \end{multicols}
\end{frame}


\section{回顾与拓展}
\subsection{回顾}
\begin{frame}
  \frametitle{回顾 | 太极拳}
  \begin{block}{太极拳健身}
打太极拳可以强壮身体,延长寿命,也就是说,打太极拳对身体健康有因果作用。但是打太极拳的人的寿命可能会与不打太极拳的人的寿命没有什么差异(或者反而打太极拳的人的寿命更短一些)。
  \end{block}
  \pause \pause \pause \pause
  \begin{block}{解析}
    可能是因为打太极拳的人都是体弱多病的人。
  \end{block}
\end{frame}

\begin{frame}
  \frametitle{回顾 | 铀矿工人}
  \begin{block}{矿工寿命}
    在铀矿工作的工人与其他人的寿命一样长(或更长),这并不能说明暴露于铀矿不会影响寿命。
  \end{block}
  \pause \pause \pause \pause
  \begin{block}{解析}
    可能是因为铀矿工人是经过挑选出来的身体健壮的人,假若当年他们不暴露于铀矿的话,寿命可能会更长一些。
  \end{block}
\end{frame}

\begin{frame}
  \frametitle{回顾 | 健康员工效应}
  \begin{block}{健康员工效应}
    有时在同一环境下,两组样本并不能直接进行比较。
  \end{block}
  \pause
  \begin{block}{实验}
假设将一组上班族与一组宇航员的健康状态进行比较研究。如果研究显示,两组没有显著差异,健康状况与工作环境之间没有相关性,我们是否就可以得出一个结论:在太空居住和工作不会给宇航员带来长期的健康风险?
  \end{block}
  \pause \pause \pause \pause
  \begin{block}{解析}
答案是不能。因为两组研究对象并没有站在同一起跑线上:宇航员团队会在申请者中挑选健康状况良好的候选人,然后按照一套综合的健康养生法进行保养,以便提前帮助宇航员克服微重力对生活带来的影响。
  \end{block}
\end{frame}

\subsection{拓展}
\begin{frame}
  \frametitle{拓展 | 四口之家的财富绝不会正好是两口之家的两倍}
  \begin{figure}
    \centering
    \includegraphics[width=0.9\textwidth]{c6.income.01.png}
  \end{figure}
\end{frame}

\begin{frame}
  \frametitle{拓展 | 不可忽视的权重}
  \begin{figure}
    \centering
    \includegraphics[width=0.8\textwidth]{c6.alone.01.png}
  \end{figure}
\end{frame}

\begin{frame}
  \frametitle{拓展 | 数字越精确结论越不可靠}
  \begin{figure}
    \centering
    \includegraphics[width=0.45\textwidth]{c6.sleep.01.png}
  \end{figure}
\end{frame}

\begin{frame}
  \frametitle{拓展 | 数字越精确结论越不可靠}
  \begin{figure}
    \centering
    \includegraphics[width=0.7\textwidth]{c6.war.01.png}
  \end{figure}
\end{frame}

\begin{frame}
  \frametitle{拓展 | 变换基数操纵百分比}
  \begin{figure}
    \centering
    \includegraphics[width=0.4\textwidth]{c6.sale.01.png}\quad
    \includegraphics[width=0.4\textwidth]{c6.sale.02.png}
  \end{figure}
\end{frame}

\begin{frame}
  \frametitle{拓展 | 不同的基期不同的结论}
  \begin{block}{问题}
让我们假设去年一夸脱牛奶值10美分,一条面包10美分。今年牛奶的价格降至5美分,而面包的价格升至20美分。现在你想证明什么呢?物价指数上升?物价指数下降?还是根本没有变化?
  \end{block}
  \begin{figure}
    \centering
    \includegraphics[width=0.9\textwidth]{c6.price.01.png}
  \end{figure}
\end{frame}

\begin{frame}
  \frametitle{拓展 | 不同的基期不同的结论}
  \begin{block}{价格上涨}
 选择去年作为基期,也就是说,以去年的价格为100\%。既然牛奶的价格降低一半(即50\%),而且面包的价格是去年的2倍(即200\%),将50\%与200\%进行平均得125\%,与去年相比,今年的价格上涨了25\%。
  \end{block}
  \begin{figure}
    \centering
    \includegraphics[width=0.7\textwidth]{c6.price.02.png}
  \end{figure}
\end{frame}

\begin{frame}
  \frametitle{拓展 | 不同的基期不同的结论}
  \begin{block}{价格下降}
 以今年的价格为基期。去年牛奶的价格是今年的200\%,而面包的价格是今年的50\%,平均数又是125\%,也就是说,去年的价格比今年的高25\%,今年的价格下降了。
  \end{block}
  \begin{figure}
    \centering
    \includegraphics[width=0.7\textwidth]{c6.price.03.png}
  \end{figure}
\end{frame}

\begin{frame}
  \frametitle{拓展 | 不同的基期不同的结论}
  \begin{block}{价格不变}
    如果你想证明价格没有发生变化,试试使用几何平均数,这时你可以随意选择基期。几何平均数不同于算术平均数或者均值,但它也是合理的计算方法,而且在某些情况下它是一种最有效的方法。计算3个数的几何平均数,只需将3个数相乘,开3次方根;4个数的几何平均数,开4次方根,以此类推。\\
    \vspace{0.5em}
  以去年为基期为例,也就是说,去年每种商品的价格都看成100\%,将两个100\%相乘再开平方根,得到100\%,这是去年价格指数的几何平均数。今年牛奶是去年的50\%,面包是去年的200\%,50\%乘以200\%得10000\%,再开平方根得100\%。价格没升也没降。
  \end{block}
\end{frame}

\begin{frame}
  \frametitle{拓展 | 将一些看似能直接相加却不能这样操作的事情加在一起}
  \begin{block}{不需要上学}
一年365天,减去三分之一即122天作为休息时间,再减去约45天作为一日三个小时的进餐时间,余下的198天中再扣除90天度暑假,21天过圣诞节和万圣节。这时余下的时间连过星期六和星期天都不够。
  \end{block}
  \pause
  \begin{block}{一年只工作一天}
    我向老板请一天假,老板推心置腹地说:“你想请一天假?看看你在向公司要求什么——一年里有365天你可以工作。一年52个星期,你已经每星期休息2天,共104天,剩下261天工作。你每天有16小时不在工作,去掉174天,剩下87天。每天你至少花30分钟时间上网,加起来每年23天,剩下64天。每天午饭时间你花掉1小时,又用掉46天,还有18天。通常你每年请2天病假,这样你的工作时间只有16天。每年有5个节假日公司休息不上班,你只干11天。每年公司还慷慨地给你10天假期,算下来你就工作1天,而你TMD还要请这一天假?”
  \end{block}
\end{frame}

\begin{frame}
  \frametitle{拓展 | 将一些看似能直接相加却不能这样操作的事情加在一起}
  \begin{figure}
    \centering
    \includegraphics[width=0.5\textwidth]{c6.year.01.png}
  \end{figure}
\end{frame}

\begin{frame}
  \frametitle{拓展 | 将一些看似能直接相加却不能这样操作的事情加在一起}
  加起来200岁的乐队,只组合一年就散伙,却拯救了整个华语乐坛!
  \vspace{-0.5em}
  \begin{figure}
    \centering
    \includegraphics[width=0.9\textwidth]{c6.music.01.jpg}
  \end{figure}
\end{frame}

\begin{frame}
  \frametitle{拓展 | 好“小”的1000万英镑}
  \begin{block}{振兴教育}
    2007年1月,英国政府大肆宣布将加拨1000万英镑的预算,“振兴小学的歌唱与音乐教育”。1000万英镑,看起来好像很大,但这个数字应该附加下列说明:全英有大约1000万名学童,几乎有一半都在念小学,将1000万英镑平均分配给500万个小学生之后,这笔预算到底可以振兴出什么结果?
  \end{block}
  \pause
  \begin{block}{托儿所}
    \begin{itemize}
      \item 5年内花费3亿英镑新增100万间托儿所,这笔钱够不够?
      \item 你找得到一周费用只有1.15英镑的托儿所吗?
    \end{itemize}
  \end{block}
  \pause
  \begin{block}{支付宝红包}
    \begin{itemize}
      \item (2017年)1.68亿人瓜分2亿五福红包——人均1.2元!
      \item (2018年)支付宝集五福全民瓜分5亿红包——2.51亿人/人均1.988元!
    \end{itemize}
  \end{block}
\end{frame}

\section{如何反驳统计资料}
\begin{frame}
  \frametitle{反驳 | 原则}
  \begin{block}{\alert{反驳统计资料}}
    怎样凭双眼就能识破虚假的统计资料,并揭开它的老底;同样重要的是,如何在这一大片充满了欺骗性的数据海洋中找出可靠有用的资料。\\
    \vspace{0.5em}
    你所接触到的统计资料,它们并非都要经受化学分析或者实验室的鉴定才能辨别真伪。但至少你可以提5个简单的问题,在寻找这些问题答案的同时,你将避免接受一些不真实的资料。
    \begin{enumerate}
      \item 谁说的?——寻找偏差(有意识的偏差和无意识的偏差)
      \item 他是如何知道的?——样本是否有偏,数值是否足够大,观察值是否足够多
      \item 遗漏了什么?——包含多少观测值,没有比较,仅给出百分数,巧妙选择基期,遗漏引起变化的原因
      \item 是否有人偷换了概念?——定义方式的改变,偷梁换柱的比较
      \item 这个资料有意义吗?——让人印象深刻的精确数据,不加控制的外推法
    \end{enumerate}
  \end{block}
\end{frame}

\begin{frame}
  \frametitle{反驳 | 数目有多大?}
  \begin{block}{把它个人化}
  太过专注于数字的“大”,往往只会造成混淆视听的效果,除非这刚好是使用这些数字的人想要达到的目标,但又往往不是,反而变成大家一再落入的陷阱。所以,在看到数字时,最重要、简单,却也最少人问起的问题,就是“这个数字大不大?”\\
  \vspace{0.5em}
  每当数字的大小,超过日常应用的熟悉范围,我们就经常会忘记以人类尺度来看待这些天文数字。然而,\alert{人类尺度是让数字变得有意义的最佳工具},也是我们每个人生下来都具备的尺度,运用起来一点也不困难。\\
  \vspace{0.5em}
  \alert{让数字变得有意义的最佳工具,就是以人类尺度来看。}
  \end{block}
\end{frame}

\begin{frame}
  \frametitle{反驳 | 数目有多大? | 实例}
  \begin{block}{阅读量}
    大学4年,借阅400本书(确切数字为476册)。
  \end{block}
  \pause
  \begin{block}{“科研人才”}
    5年发表40余篇科研论文!——灌水!
  \end{block}
  \pause
  \begin{block}{学科评估}
    2017年,天津财经大学,5名评估专家,5天的时间,“评阅”4000多份试卷!(工作到晚上10点,给准备夜宵,……)
  \end{block}
\end{frame}

\section{寄语}
\begin{frame}
  \frametitle{你的世界别人不懂}
  \begin{figure}
    \centering
    \includegraphics[width=0.5\textwidth]{c6.hope.01.jpg}
  \end{figure}
\end{frame}

\begin{frame}
  \frametitle{你的时区独一无二}
  \begin{figure}
    \centering
    \includegraphics[width=0.5\textwidth]{c6.hope.02.jpg}
  \end{figure}
\end{frame}

\begin{frame}
  \frametitle{你的时区独一无二}
  \begin{block}{时区(1/2)}
纽约时间比加州时间早三个小时,New York is 3 hours ahead of California,\\
但加州时间并没有变慢。but it does not make California slow.\\
有人22岁就毕业了,Someone graduated at the age of 22,\\
但等了五年才找到稳定的工作!but waited 5 years before securing a good job!\\
有人25岁就当上CEO,却在50岁去世。Someone became a CEO at 25, and died at 50.\\
也有人迟到50岁才当上CEO,然后活到90岁。While another became a CEO at 50, and lived to 90 years.\\
有人单身,同时也有人已婚。Someone is still single, while someone else got married.\\
奥巴马55岁就退休,川普70岁才开始当总统。Obama retires at 55, but Trump starts at 70.\\
  \end{block}
\end{frame}

\begin{frame}
  \frametitle{你的时区独一无二}
  \begin{block}{时区(2/2)}
    {\small
世上每个人本来就有自己的发展时区。Absolutely everyone in this world works based on their Time Zone.\\
身边有些人看似走在你前面,也有人看似走在你后面。People around you might seem to go ahead of you, some might seem to be behind you.\\
但其实每个人在自己的时区有自己的步程。But everyone is running their own RACE, in their own TIME.\\
不用嫉妒或嘲笑他们。Don't envy them or mock them.\\
他们都在自己的时区里,你也是!They are in their TIME ZONE, and you are in yours!\\
生命就是等待正确的行动时机。Life is about waiting for the right moment to act.\\
所以,放轻松。So, RELAX.\\
你没有落后。You're not LATE.\\
你没有领先。You're not EARLY.\\
在你自己的时区里,一切安排都准时。You are very much ON TIME, and in your TIME ZONE.
}
  \end{block}
\end{frame}



\input{snippet/class_tail}
